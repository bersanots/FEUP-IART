\documentclass[conference]{IEEEtran}
\IEEEoverridecommandlockouts
% The preceding line is only needed to identify funding in the first footnote. If that is unneeded, please comment it out.
\usepackage{cite}
\usepackage{amsmath,amssymb,amsfonts}
\usepackage{algorithmic}
\usepackage{graphicx}
\usepackage{textcomp}
\usepackage{xcolor}
\def\BibTeX{{\rm B\kern-.05em{\sc i\kern-.025em b}\kern-.08em
    T\kern-.1667em\lower.7ex\hbox{E}\kern-.125emX}}
\begin{document}

\title{Resolução do Klotski utilizando Métodos de Pesquisa em Python (Tema 1/ Grupo 31)\\
{\footnotesize \textsuperscript{*}Note: Sub-titles are not captured in Xplore and
should not be used}
\thanks{Identify applicable funding agency here. If none, delete this.}
}

\author{\IEEEauthorblockN{André Lopes dos Santos (200505634)}
\IEEEauthorblockA{\textit{Departamento de Engenharia Informática} \\
\textit{Faculdade de Engenharia da Universidade do Porto}\\
Porto, Portugal \\
up200505634@fe.up.pt}
\and
\IEEEauthorblockN{Bernardo Oliveira Teixeira Santos (201504711)}
\IEEEauthorblockA{\textit{Departamento de Engenharia Informática} \\
\textit{Faculdade de Engenharia da Universidade do Porto}\\
Porto, Portugal \\
up201504711@fe.up.pt}
\and
\IEEEauthorblockN{Miguel Rossi  Seabra (200604224)}
\IEEEauthorblockA{\textit{Departamento de Engenharia Informática} \\
\textit{Faculdade de Engenharia da Universidade do Porto}\\
Porto, Portugal \\
ei06054@fe.up.pt}
}

\maketitle

\begin{abstract}
Abordagem a diversos métodos de pesquisa e seus algoritmos, que são uma componente importante da inteligência artificial, assim como à utilização dos mesmos na resolução do Klotski, um conhecido puzzle de deslizamento de blocos.
A linguagem utilizada para a implementação do mesmo é o Python, sendo que o código nesta linguagem para a implementação deste puzzle e para a aplicação dos diferentes algoritmos é adaptado para atingir os objetivos.
\end{abstract}

\begin{IEEEkeywords}
Inteligência Artificial, Pesquisa, Algoritmo A*, Klotski
\end{IEEEkeywords}

\section{Introdução}
Neste trabalho abordar-se-ão diversos métodos de pesquisa em Python aplicados à resolução de puzzles de blocos conhecidos pelo nome “Klotski”, mas mais especificamente a iteração deste que está presente na Google Play Store **INSERIR REFERENCIA** Discutir-se-á a história do jogo, as suas regras e a formulação formal do problema. Também falar-se-á dos diferentes métodos de pesquisa, da sua implementação em Python e a eficiência dos mesmos para o jogo em questão. 

\section{Descrição do Problema}
Os puzzles Klotski são puzzles de movimento de blocos em que o objetivo é mover um bloco específico para um sítio específico, movendo todas as outras peças nesse processo. Na Figura 1 pode-se ver um exemplo de um puzzle.

INSERIR IMAGEM

A origem deste puzzle é ainda incerta, mas pensa-se que uma primeira iteração do mesmo tenha sido patenteada por Henry Walton em 1893, embora existam diversos países (Inglaterra, Japão) que reclamam a autoria do puzzle “Original”, sendo que ainda hoje é desconhecida (ou não existe consenso) quanto à origem do mesmo. [2]
A variante aqui estudada é a iteração presente na Google Play Store [1] que acrescenta regras como blocos que não se podem mover (i.e. paredes) e outros blocos azuis que se portam como paredes até que o bloco vermelho toque em todos eles, fazendo-os desaparecer. Na Figura 2 pode-se ver um exemplo de um puzzle que contém paredes (representadas por blocos castanhos) e os já mencionados blocos azuis.


INSERIR IMAGEM



\section{Formulação do Problema}
Representação do Estado: O estado é representado por uma Matriz de tamanho variável (a * b) [0 – espaço livre], pela posição da forma geométrica vermelha - [2] -, pelo local das marcas vermelhas - [1] -, pelas outras peças móveis (cada uma terá um número associado para as distinguir entre si, exceto aqueles números já reservados para outras peças), pelos locais das paredes fixas (caso estas existam - [X]) e pelos locais das paredes azuis que desaparecem depois da peça vermelha entrar em contacto com todas elas - [Y].
Estados Iniciais: Os estados iniciais variam conforme o puzzle a ser resolvido. Na Figura 1 e na Figura 2 podem-se ver dois exemplos de estados iniciais de dois puzzles distintos.

Teste Objetivo: O bloco vermelho estar na posição final (no local das pequenas marcas vermelhas) e chegar lá no número mínimo de jogadas possível.
Operadores/Jogadas: Mov(N,M,E) – Movimentação da peça N (sendo N um número acima de 2 mas igual ou inferior ao número mais alto), M casas no eixo E (x ou y).
Pré-condições: todos os elementos da peça possam mover-se na direção desejada:
• Quando Mov(a,-3,y) para todos os elementos de a[x][y], nas 3 casas inferiores [y-1][y-2][y-3] o elemento presente terá de ser (a,0,1).
Efeitos: Mov(a,-3,y) – todas as células da matriz com valor ‘a’, passarão das coordenadas a[x][y] para a[x][y-3]



\section{Trabalho Relacionado}
Foram encontradas diversas implementações em Python do puzzle Klotski, algumas com solvers já implementados [3] [4] [5] [6] [7]. A implementação do Klotski será baseada nestas mesmas fontes, sendo que serão realizadas diversas adaptações ao caso específico deste trabalho.
Os métodos de pesquisa utilizados serão pesquisa em largura, pesquisa em profundidade, aprofundamento progressivo, custo uniforme, pesquisa gulosa e Algoritmo A*, sendo a implementação destes métodos baseados naqueles presentes no livro “Artificial Intelligence: A Modern Approach” (sendo este o livro principal da cadeira de Inteligência Artificial) [8], sendo que o código (em várias linguagens) para a implementação destes algoritmos está disponível online [9].

\section{Implementação do Jogo}
Lorem ipsum dolor sit amet, consectetur adipiscing elit, sed do eiusmod tempor incididunt ut labore et dolore magna aliqua. Ut enim ad minim veniam, quis nostrud exercitation ullamco laboris nisi ut aliquip ex ea commodo consequat. Duis aute irure dolor in reprehenderit in voluptate velit esse cillum dolore eu fugiat nulla pariatur. Excepteur sint occaecat cupidatat non proident, sunt in culpa qui officia deserunt mollit anim id est laborum.

\section{Algoritmos de Pesquisa}

Lorem ipsum dolor sit amet, consectetur adipiscing elit, sed do eiusmod tempor incididunt ut labore et dolore magna aliqua. Ut enim ad minim veniam, quis nostrud exercitation ullamco laboris nisi ut aliquip ex ea commodo consequat. Duis aute irure dolor in reprehenderit in voluptate velit esse cillum dolore eu fugiat nulla pariatur. Excepteur sint occaecat cupidatat non proident, sunt in culpa qui officia deserunt mollit anim id est laborum.
\section{Experiências e Resultados}
Lorem ipsum dolor sit amet, consectetur adipiscing elit, sed do eiusmod tempor incididunt ut labore et dolore magna aliqua. Ut enim ad minim veniam, quis nostrud exercitation ullamco laboris nisi ut aliquip ex ea commodo consequat. Duis aute irure dolor in reprehenderit in voluptate velit esse cillum dolore eu fugiat nulla pariatur. Excepteur sint occaecat cupidatat non proident, sunt in culpa qui officia deserunt mollit anim id est laborum.
\section{Conclusões e Perspetivas de Desenvolvimento}
Lorem ipsum dolor sit amet, consectetur adipiscing elit, sed do eiusmod tempor incididunt ut labore et dolore magna aliqua. Ut enim ad minim veniam, quis nostrud exercitation ullamco laboris nisi ut aliquip ex ea commodo consequat. Duis aute irure dolor in reprehenderit in voluptate velit esse cillum dolore eu fugiat nulla pariatur. Excepteur sint occaecat cupidatat non proident, sunt in culpa qui officia deserunt mollit anim id est laborum.






\section*{Referências Bibliográficas}


Please number citations consecutively within brackets \cite{b1}. The 
sentence punctuation follows the bracket \cite{b2}. Refer simply to the reference 
number, as in \cite{b3}---do not use ``Ref. \cite{b3}'' or ``reference \cite{b3}'' except at 
the beginning of a sentence: ``Reference \cite{b3} was the first $\ldots$''

Number footnotes separately in superscripts. Place the actual footnote at 
the bottom of the column in which it was cited. Do not put footnotes in the 
abstract or reference list. Use letters for table footnotes.

Unless there are six authors or more give all authors' names; do not use 
``et al.''. Papers that have not been published, even if they have been 
submitted for publication, should be cited as ``unpublished'' \cite{b4}. Papers 
that have been accepted for publication should be cited as ``in press'' \cite{b5}. 
Capitalize only the first word in a paper title, except for proper nouns and 
element symbols.

For papers published in translation journals, please give the English 
citation first, followed by the original foreign-language citation \cite{b6}.

\begin{thebibliography}{00}
\bibitem{b1} G. Eason, B. Noble, and I. N. Sneddon, ``On certain integrals of Lipschitz-Hankel type involving products of Bessel functions,'' Phil. Trans. Roy. Soc. London, vol. A247, pp. 529--551, April 1955.
\bibitem{b2} J. Clerk Maxwell, A Treatise on Electricity and Magnetism, 3rd ed., vol. 2. Oxford: Clarendon, 1892, pp.68--73.
\bibitem{b3} I. S. Jacobs and C. P. Bean, ``Fine particles, thin films and exchange anisotropy,'' in Magnetism, vol. III, G. T. Rado and H. Suhl, Eds. New York: Academic, 1963, pp. 271--350.
\bibitem{b4} K. Elissa, ``Title of paper if known,'' unpublished.
\bibitem{b5} R. Nicole, ``Title of paper with only first word capitalized,'' J. Name Stand. Abbrev., in press.
\bibitem{b6} Y. Yorozu, M. Hirano, K. Oka, and Y. Tagawa, ``Electron spectroscopy studies on magneto-optical media and plastic substrate interface,'' IEEE Transl. J. Magn. Japan, vol. 2, pp. 740--741, August 1987 [Digests 9th Annual Conf. Magnetics Japan, p. 301, 1982].
\bibitem{b7} M. Young, The Technical Writer's Handbook. Mill Valley, CA: University Science, 1989.
\end{thebibliography}
\vspace{12pt}
\color{red}
IEEE conference templates contain guidance text for composing and formatting conference papers. Please ensure that all template text is removed from your conference paper prior to submission to the conference. Failure to remove the template text from your paper may result in your paper not being published.

\end{document}
